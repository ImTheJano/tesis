\section{Ecuaci\'on de Schr\"odinger}
Erwin Schr\"odinger postul\'o que la ecuaci\'on para una onda $\Psi(x,t) $
deb\'ia ser, en el caso estacionario, es decir cuando no hay dependencia del
tiempo

\begin{equation*}
	H(x, \widehat{p}) \Psi(x,t) = E \Psi(x,t), \ \ \ 
	\widehat{p}=-i\hbar\frac{\partial}{\partial x '}
\end{equation*}

donde $E$ era la energ\'ia del sistema asociado a la onda $\Psi$. Es decir, en el
caso cuando tenemos un Hamiltoniano est\'andar,

\begin{equation*}
	H(x,p)=\frac{p^2}{2m}+V(x)
\end{equation*}

siendo $V(x)$ la funci\'on potencial (energ\'ia potencial), se tiene la ecuaci\'on de diferencial

\begin{equation}
	-\frac{\hbar^2}{2m} \frac{\partial^2}{\partial x^2} + V(x) \Psi(x,t) = E\Psi(x,t)
	\label{schrodinger_eq_3}
\end{equation}

Una de las pruebas de fuego de su ecuaci\'on fue el caso $V(x)=0$, es decir cuando se ten\'ia el
movimiento de una part\'icula libre. Si consideramos el caso unidimensional, y hacemos $V (x) = 0$,
la soluci\'on deb\'ia ser una onda plana del tipo

\begin{equation*}
	\Psi(x,t)=A cos(kx+wt), \ \ \ 
	k=\frac{2\pi}{\lambda}
\end{equation*}

Si sustituimos este valor en la ecuaci\'on de Schr\"odinger tenemos el valor

\begin{equation*}
	E=\frac{h^2 k^2}{2m}
\end{equation*}

que igualado con el valor de la energ\'ia cin\'etica --recordemos que $V(x)=0$-- nos da

\begin{equation*}
	p=\hbar k = \frac{2 \pi \hbar}{\lambda}
\end{equation*}

que justamente era la f\'ormula que hab\'ia postulado De Broglie.

Pero su mayor \'exito estaba por llegar. Schr\"odinger decidi\'o aplicar su ecuaci\'on al \'atomo de
hidr\'ogeno. Como en este caso el potencial era

\begin{equation*}
	V(r)=-\frac{e^{2}}{r}, \ \ \ 
	r=\sqrt{x^2+y^2+z^2}
\end{equation*}

obtuvo la ecuaci\'on

\begin{equation}
	-\frac{\hbar^2}{-2m}
	\left( \frac{\partial^2}{\partial x^2} \frac{\partial^2}{\partial y^2} \frac{\partial^2}{\partial z^2} \right)
	\Psi(x,y,z)-\frac{e^{2}}{r} \Psi(x,y,z)= E \Psi(x,y,z)
	\label{schrodinger_eq_8}
\end{equation}

A esta ecuaci\'on volveremos m\'as adelante. Lo importante era que Schr\"odinger sab\'ia como tratar
este tipo de ecuaciones y la resolvi\'o. Primero la escribi\'o en coordenadas esf\'ericas y luego aplic\'o la
separaci\'on de variables. La parte angular del laplaciano, $ \Delta := \frac{\partial^2}{\partial x^2} \frac{\partial^2}{\partial y^2} \frac{\partial^2}{\partial z^2} $ en coordenadas
esf\'ericas era muy sencilla de resolver apareciendo las funciones o arm\'onicos esf\'ericos de Laplace.
En particular Schr\"odinger obtuvo para los valores de la energ\'ia en el estado estacionario del
\'atomo de hidr\'ogeno la f\'ormula

\begin{equation*}
	E_n = -\frac{me^4}{2\hbar}\frac{1}{n^2}
\end{equation*}

que era la misma de Bohr

Finalmente Schr\"odinger, igual que hizo Heisenberg, <<dedujo>> una ecuaci\'on para la din\'amica
de un sistema, que en el caso unidimensional tiene la forma

\begin{equation*}
	H \left( x,-i \hbar \frac{\partial}{\partial x} \right) \Psi(x,t) =
	i\hbar \frac{\partial}{\partial t} \Psi(x,t)
\end{equation*}

Si tenemos un Hamiltoniano est\'andar, \'esta se transforma en

\begin{equation*}
	-\frac{\hbar^2}{2m} \frac{\partial^2}{\partial x^2} \Psi(x,t) + V(x) \Psi(x,t)=
	i\hbar \frac{\partial}{\partial t} \Psi(x,t)
\end{equation*}

En particular, de \ref{schrodinger_eq_3} se pod\'ia deducir f\'acilmente la ecuaci\'on 
\ref{schrodinger_eq_8}, para los sistemas, introduciendo la factorizaci\'on

\begin{equation*}
	\Psi(x,t)=\Psi(x) exp \left( -\frac{iEt}{\hbar} \right)
\end{equation*}