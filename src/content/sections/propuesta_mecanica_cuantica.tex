\section{Mec\'anica cu\'antica}

Se considera la aplicaci\'on de AG y RNA a un c\'alculo mec\'anico
cu\'antico. El sistema cu\'antico unidimensional (1D) se rige por
la ecuaci\'on de Schr\"dinger

\begin{equation}
	\widehat{H}\Psi(x)=\left(-\frac{\hbar^2}{2\mu}\nabla^2+V(x)\right)\Psi(x)=E\Psi(x)
	\label{schrodinger}
\end{equation}

Donde, $\mu$ y $V(x)$ representan la masa de la part\'icula y el
potencial externo, $\widehat{H}$, $\Psi(x)$ y $E$ representan al sistema
Hamiltoniano, la funci\'on propia y el valor propio respectivamente.
Antes de utilizar la RNA y el AG, necesitamos hacer cumplir la correspondencia
uno a uno entre las diferentes representaciones de la soluci\'on; la funci\'on
propia, la red neuronal y la cadena del algoritmo gen\'etico. Para iniciar la b\'usqueda de soluci\'on
global por AG, representamos la funci\'on de prueba en forma de cadena de AG.
Por otro lado, nuestro objetivo es aplicar el optimizador determinista bajo
la representaci\'on RNA para refinar la soluci\'on. La \ref{fig:nnGaWf} describe
esquem\'aticamente la relaci\'on entre esas tres representaciones diferentes.

La soluci\'on de (\ref{schrodinger}) es representada por la funci\'on
propia de prueba, la red neuronal artificial y el algoritmo
gen\'etico. La b\'usqueda de la soluci\'on se lleva a cabo por el
AG, entonces la soluci\'on o funci\'on de prueba se codifica en
un cromosoma. Se lleva a cabo una optimizaci\'on determinista
sobre la RNA con objetivo de afinar la soluci\'on. La identidad
de un estado cu\'antico se representa por una funci\'on de onda
en el correspondiente espacio vectorial. En general la funci\'on
de onda $\psi(x)$ se expresa como: 

\begin{equation}
	\Psi(x)=A(x)sen[S(x)]\label{waveFunc}
\end{equation}

Donde $A(x)$ es la amplitud de onda y $S(x)$ es la fase de la
funci\'on onda. La representaci\'on por medio de la RNA recibe
como estimulo la coordenada $x$ y responde con dos valores
de salida cada uno asociado con $A(x)$ y $B(x)$. En t\'erminos
de (\ref{output}) y (\ref{stimulus}), la relaci\'on entre la RNA y la funci\'on de onda
se expresa como $o_1=A(x)$, $o_2=S(x)$ y $n_1=x$. \ref{fig:nnGaWf}.

\begin{figure}[H]
	\centering
	\includegraphics[width=7cm]{img/fig_nnGaWf.png}
	\caption{Tres representaciones diferentes de la identidad del estado cu\'antico.}
	\label{fig:nnGaWf}
\end{figure}

Aunque nos hemos limitado al sistema 1D por simplicidad, el enfoque
actual puede expandirse f\'acilmente al sistema multidimensional. Para
el problema N-dimensional, el perceptr\'on con $N_i = N_k$, $NK = 2$ puede
usarse para representar la funci\'on de onda $\phi(x_1,x_2,x_3...x_n)$.
La tercera representaci\'on, cadena del AG se define a trav\'es de los par\'ametros
RNA. Dado que un conjunto de par\'ametros dado de la RNA \'unicamente
corresponde a una funci\'on de onda espec\'ifica, la cadena AG generada
al codificar estos par\'ametros RNA tambi\'en representa de manera \'unica
la misma identidad del estado cu\'antico. Tenga en cuenta que el valor
propio $E$ se codifica en la cadena AG junto con los par\'ametros de red
como se muestra en \ref{fig:nnGaWf}.
Ahora, describiremos c\'omo resolver la ecuaci\'on de Schr\"odinger usando
estas representaciones. Para ubicar una soluci\'on aproximada en el vasto
espacio de b\'usqueda, utilizamos el AG como pre-optimizador. Es necesario
para definir el puntaje de aptitud f\'isica $f$ para usar el AG. Para este
prop\'osito, definimos la cantidad que refleja el error contenido en la
funci\'on de prueba $\phi(x)$

\begin{equation}
	R=\frac{\left\langle\Psi|(\hat{H}-E)^2|\Psi\right\rangle}{\left\langle\Psi|\Psi\right\rangle}=\frac{\int|(\hat{H}-E)|\Psi(x)|^2 dx}{\int|\Psi(x)|^2 dx}
	\label{error}
\end{equation}

Por \'ultimo, la representaci\'on con el AG se define con base en
los par\'ametros de la RNA. Cada conjunto de par\'ametros de
la RNA corresponde con una funci\'on de onda especifica, en
el cromosoma del AG se codifican estos par\'ametros que representan
la identidad \'unica de un estado cu\'antico. El valor
propio $E$ se codifica en el cromosoma junto con los par\'ametros
de la red, \'este identifica a cada conjunto de par\'ametros y
por ende a cada funci\'on de onda.

Para localizar una soluci\'on aproximada en el espacio de
b\'usqueda, el AG es utilizado como pre-optimizador, para ello
es necesario determinar la funci\'on de aptitud $f$, la cual
representa el medio ambiente dentro del AG. Para ello se define la
cantidad que refleja el error contenido en la funci\'on de prueba
$\Psi(x)$

La funci\'on propia que satisface (\ref{schrodinger}) se asocia con $R=0$.
La funci\'on de aptitud para el problema de optimizaci\'on relacionado
con el algoritmo gen\'etico, y que se busca maximizar se expresa como:

\begin{equation}
	f=e^{-R}
\end{equation}

El resolver el problema de optimizaci\'on con el AG,
implica encontrar la funci\'on propia y su correspondiente valor
propio, esto implica encontrar una soluci\'on de la ecuaci\'on
(\ref{schrodinger}). El operador $\nabla^2$ que aparece en el Hamiltoniano puede
ser evaluado por la ecuaci\'on (\ref{lamda}).

Para mejorar la eficiencia de b\'usqueda en las primeras generaciones,
introducimos el m\'etodo de evaluaci\'on de puntos aleatorios (MEPA) y
redefinimos la cantidad $\tilde{R}$ como

\begin{equation}
	\tilde{R}=\frac{ \sum^{M}_{x_i} \left| \left(\hat{H}-E\right) \psi \left(x_i\right) \right| ^2 }{ \sum^{M}_{x_i} \left| \psi(x_i) \right| ^2 }
	\label{rpemEquation}
\end{equation}

puntaje de aptitud como

\begin{equation}
	\tilde{f}=e^{-\tilde{R}}
	\label{fitnessScoreEquation}
\end{equation}

Aqu\'i, $x_i$ representa los puntos distribuidos aleatoriamente sobre
el espacio de coordenadas de posici\'on y $M$ denota el n\'umero total de
puntos aleatorios (PA) introducidos aqu\'i.

El conjunto de RP com\'un se utiliza para evaluar las cadenas que pertenecen
a la misma generaci\'on y el conjunto de RP se renueva a medida que avanza la
generaci\'on. Cabe se\~nalar que la puntuaci\'on de aptitud f\'isica $\tilde{f}$ para la
cadena id\'entica puede tener un valor diferente en diferentes generaciones.
Los efectos del MEPA sobre la convergencia de la puntuaci\'on de aptitud en las primeras
generaciones se analizar\'an en las secciones siguientes. La soluci\'on aproximada obtenida
por la mejora gen\'etica del AG debe refinarse a trav\'es del entrenamiento de
la red o el proceso de aprendizaje. Para una red de tipo perceptr\'on, es
ampliamente conocido que la red puede ser entrenada por el procedimiento
llamado Backpropagation, en el que uno hace que las se\~nales de error
se propaguen hacia atr\'as para modificar los pesos y los par\'ametros de umbral.

Dado que las derivadas de la salida de la red con respecto a esos par\'ametros
RNA se conocen como Eqs. \ref{derivativeOfO_kRespectW_ij} - \ref{derivativeOfO_kRespectTheta_k},
la derivada de $R$ se puede obtener a trav\'es de la definici\'on Eq. \ref{error}
Usando esas informaciones derivadas, se puede emplear un optimizador determinista
general para minimizar $R$, como los enfoques basados en gradientes ampliamente
utilizados. En el presente estudio, adoptamos el m\'etodo de descenso m\'as
pronunciado para obtener la soluci\'on final.