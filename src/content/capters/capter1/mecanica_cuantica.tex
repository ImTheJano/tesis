\subsection{Mec\'anica cu\'antica}

Sinergia entre redes neuronales artificiales, algoritmo
gen\'eticos y mec\'anica cu\'antica. Un sistema cu\'antico
uno-dimensional es gobernado por la ecuaci\'on de Schr\"odinger:

\begin{equation}
	\widehat{H}\Psi(x)=(-\frac{\hbar^2}{2\mu}\nabla^2+V(x))\Psi(x)=E\Psi(x)
	\label{schrodinger}
\end{equation}

Donde, $\mu$ y $V(x)$ representan la masa de la part\'icula y el
potencial externo, $\widehat{H}$, $\Psi(x)$ y $E$ representan al sistema
Hamiltoniano, la funci\'on propia y el valor propio respectivamente.
La soluci\'on de (\ref{schrodinger}) es representada por la funci\'on
propia de prueba, la red neuronal artificial y el algoritmo
gen\'etico. La b\'usqueda de la soluci\'on se lleva a cabo por el
AG, entonces la soluci\'on o funci\'on de prueba se codifica en
un cromosoma. Se lleva a cabo una optimizaci\'on determinista
sobre la RNA con objetivo de afinar la soluci\'on. La identidad
de un estado cu\'antico se representa por una funci\'on de onda
en el correspondiente espacio vectorial. En general la funci\'on
de onda $\psi(x)$ se expresa como: 

\begin{equation}
	\Psi(x)=A(x)sen[S(x)]\label{fonda}
\end{equation}

Donde $A(x)$ es la amplitud de onda y $S(x)$ es la fase de la
funci\'on onda. La representaci\'on por medio de la RNA recibe
como estimulo la coordenada $x$ y responde con dos valores
de salida cada uno asociado con $A(x)$ y $B(x)$. En t\'erminos
de (\ref{salida}) y (\ref{estimulo}), la relaci\'on entre la RNA y la funci\'on de onda
se expresa como $o_1=A(x)$, $o_2=S(x)$ y $n_1=x$. Por
\'ultimo, la representaci\'on con el AG se define con base en
los par\'ametros de la RNA. Cada conjunto de par\'ametros de
la RNA corresponde con una funci\'on de onda especifica, en
el cromosoma del AG se codifican estos par\'ametros que representan
la identidad \'unica de un estado cu\'antico. El valor
propio $E$ se codifica en el cromosoma junto con los par\'ametros
de la red, \'este identifica a cada conjunto de par\'ametros y
por ende a cada funci\'on de onda.

\begin{figure}[H]
	\centering
	\includegraphics[width=7cm]{img/fig_nnGaWf.png}
	\caption{Diferente representaci\'on de identidad del estado cuantico}
	\label{fig:nnGaWf}
\end{figure}

Para localizar una soluci\'on aproximada en el espacio de
b\'usqueda, el AG es utilizado como pre-optimizador, para ello
es necesario determinar la funci\'on de aptitud $f$, la cual
representa el medio ambiente dentro del AG. Para ello se define la
cantidad que refleja el error contenido en la funci\'on de prueba
$\Psi(x)$:
	
\begin{equation}
	R=\frac{\left\langle\Psi|(\hat{H}-E)^2|\Psi\right\rangle}{\left\langle\Psi|\Psi\right\rangle}=\frac{\int|(\hat{H}-E)|\Psi(x)|^2 dx}{\int|\Psi(x)|^2 dx}
	\label{error}
\end{equation}    

La funci\'on propia que satisface (\ref{schrodinger}) se asocia con $R=0$.
La funci\'on de aptitud para el problema de optimizaci\'on relacionado
con el algoritmo gen\'etico, y que se busca maximizar se expresa como:

\begin{equation}
	f=e^{-R}
\end{equation}

El resolver el problema de optimizaci\'on con el AG,
implica encontrar la funci\'on propia y su correspondiente valor
propio, esto implica encontrar una soluci\'on de la ecuaci\'on
(\ref{schrodinger}). El operador $\nabla^2$ que aparece en el Hamiltoniano puede
ser evaluado por la ecuaci\'on (\ref{lamda}).
