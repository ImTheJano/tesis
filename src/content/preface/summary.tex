\prefacesection{Resumen}
En este documento se presenta un nuevo enfoque para resolver la ecuaci\'on de
Schr\"odinger basada en el algoritmo gen\'etico (AG) y la red
neuronal artificial (RN). La red de tipo perceptr\'on de
avance se usa para representar la funci\'on, mientras que
los par\'ametros de red se optimizan mediante un algoritmo
microgen\'etico para que el RN satisfaga la ecuaci\'on de
Schr\"odinger.
En el proceso de mejoramiento de AG, se introduce el
m\'etodo de evaluaci\'on de puntos aleatorios (MEPA) para la
evaluaci\'on de aptitud para mejorar la convergencia. La
soluci\'on final se obtiene invocando un optimizador
determinista que corresponde a un <<proceso de aprendizaje>>
de la RN. El presente m\'etodo se prueba en el c\'alculo del
oscilador arm\'onico unidimensional y otros potenciales del
modelo.