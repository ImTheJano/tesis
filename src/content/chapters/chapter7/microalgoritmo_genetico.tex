\subsection{Algoritmo gen\'etico}

Un algoritmo gen\'etico (AG) requiere codificar las variables a
ser optimizadas para trabajar. Estas variables forman
una cadena, idea correspondiente con el cromosoma en biolog\'ia.
Cada variable real se representa como una subcadena
binaria, parte de la cadena final o cromosoma.

El azar constituye el elemento conductor de la fuerza de
un AG que lo hace un optimizador eficiente. El operador mutaci\'on
es una fuente de azar en los algoritmos gen\'eticos cl\'asicos.
Una propuesta distinta conocida como micro-AG trata
de restablecer ocasionalmente la poblaci\'on en el proceso de
b\'usqueda global, por tal efecto se requiere de una poblaci\'on
peque\~na.

El AG comienza con la construcci\'on de una poblaci\'on
inicial aleatoria con una longitud de cromosoma dependiente
de la cantidad de par\'ametros a ser optimizados. El paso
siguiente consiste en evaluar a cada individuo de la poblaci\'on
con lo que se le asigna un valor de aptitud. La selecci\'on simula
el proceso de selecci\'on natural donde un individuo con alto
puntaje de aptitud se reproduce, y aquel individuo con puntajes
pobres es eliminado. Aqu\'i se considera la selecci\'on por
torneo, donde aleatoriamente se eligen dos individuos de la
poblaci\'on, sus puntajes de aptitud son comparados, aquel con
mayor valor se mantiene para la siguiente etapa, este proceso
se repite hasta tener un n\'umero considerable de individuos.

La poblaci\'on se actualiza por operaciones de apareamiento
y cruzamiento. La operaci\'on de cruzamiento lleva a cabo
el intercambio aleatorio de informaci\'on gen\'etica entre pares
de individuos para crear a sus descendientes. Se emplea cruzamiento
uniforme, en el cual se consideran diversos puntos
de cruzamiento determinados aleatoriamente en conjunci\'on
con sus respectivas posiciones en el cromosoma.

Por \'ultimo, en cada generaci\'on se verifica la convergencia
de la poblaci\'on. \'esta se mide por $D$ que define las diferencias
entre el individuo m\'as apto con el resto de los individuos
de la poblaci\'on. Se\~nal que la poblaci\'on evoluciona es que
$D\geq\epsilon\geq0.05$ en tal caso se pasa a la siguiente generaci\'on.
En otro caso se aplica elitismo, esto significa dejar al individuo
m\'as apto, y restablecer la poblaci\'on aleatoriamente.

\begin{figure}[H]
	\centering
	\includegraphics[width=10cm]{img/fig_geneticAlgorithm.png}
	\caption{Algoritmo genetico.}
	\label{fig:geneticAlgorithm}
\end{figure}
