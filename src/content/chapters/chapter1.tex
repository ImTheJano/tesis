\chapter{Introducci\'on}
	\section{Antecedentes}

	\section{Justificac\'on}
	La construcci\'on de sistemas inteligentes donde se presenta la sinergia entre
	redes neuronales y programaci\'on evolutiva con el fin de resolver ecuaciones
	diferenciales constituye un modesto paso en la construcci\'on de la llamada 
	m\'aquina universal, es decir, aquella que tendr\'a  la capacidad de resolver 
	cualquier problema. El aporte de este trabajo es el manejo de un modelo donde
	concurren la inteligencia artificial y la mec\'anica cu\'antica, dado que se busca 
	que este sistema aprenda a resolver la ecuaci\'on de Shr\"odinger, la cual 
	se asemeja con un oscilador armonico simple. Este trabajo podr\'a  marcar el
	inicio para resolver ecuaciones de la fisicamatematica por medio de t\'ecnicas 
	de la IA.
	
	\section{Objetivos}
	\subsection{Objetivo principal}
	Proponer un modelo computacional que involucre a las redes neuronales artificiales
	con los algoritmos gen\'eticos para resolver la ecuaci\'on de Shr\"odinger.
	\subsection{Objetivos espec\'ificos}
	\begin{itemize}
		\item Compilar, seleccionar y extraer informaci\'on acerca de:
		\begin{itemize}
			\item Redes neuronales
			\item Algoritmos gen\'eticos, y
			\item Principios de mec\'anica cu\'antica
		\end{itemize}
		\item Comprender la interacci\'ion entre redes neuronales y algoritmos geneticos
		\item Entender la matematica de un sistema cu\'antico descrito por la ecuaci\'on de Sch\"odinger
		\item Contruir el algoritmos
		\item Generar reusultados
	\end{itemize}
	\section{Contribuciones}
	
	\section{Organizaci\'on del documento}
	\subsection{Cap\'itulo 1}
	\subsection{Cap\'itulo 2}
	\subsection{Cap\'itulo 3}
	\subsection{Cap\'itulo 4}
	\subsection{Cap\'itulo 5}
	\subsection{Cap\'itulo 6}