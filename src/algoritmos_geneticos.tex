\section{Algoritmos geneticos}
La computaci\'on evolutiva es la rama de la inteligencia artificial que engloba
a todas aquellas t\'ecnicas de resoluci\'on de problemas basadas en la evoluci\'on
de las especies y la supervivencia del m\'as apto.
Dentro de ella encontramos a los algoritmos gen\'eticos (genetic algorithms), las
estrategias evolutivas (evolution strategies) y la programaci\'on evolutiva
(evolutionary programming) entre otros.
Las t\'ecnicas evolutivas han sido aplicadas con \'exito a distintos tipos de
problemas como optimizaci\'on de par\'ametros, planificaci\'on de tareas, dise\~no,
etc.
Estos algoritmos codifican las posibles soluciones en estructuras llamadas
cromosomas (o individuos). A un conjunto de individuos se lo conoce como
poblaci\'on y representan un conjunto de soluciones posibles al problema.
Mediante la aplicaci\'on de un conjunto de operadores
gen\'eticos sobre la poblaci\'on se va refinando gradualmente la soluci\'on hasta
alcanzar un resultado que cumpla con las condiciones requeridas.
El primer paso en la aplicaci\'on de un algoritmo gen\'etico consiste en la
generaci\'on de una poblaci\'on inicial. En general esta poblaci\'on se genera de
manera aleatoria, y el tama\~no de dicha poblaci\'on (la cantidad de individuos que
la compone) es un par\'ametro que se define durante el dise\~no del algoritmo
gen\'etico. Una vez generada esta poblaci\'on se debe evaluar la aptitud (fitness)
de cada individuo.
El operador de selecci\'on es el encargado de decidir cuales individuos
contribuir\'an en la formaci\'on de la pr\'oxima generaci\'on de individuos. Este
mecanismo simula el proceso de selecci\'on natural, mediante el cual s\'olo los
individuos m\'as adaptados al ambiente se reproducen. El
mecanismo de selecci\'on forma una poblaci\'on intermedia, que esta compuesta
por los individuos con mayor aptitud de la generaci\'on actual.
La siguiente fase del algoritmo consiste en la aplicaci\'on de los operadores
gen\'eticos. El primero de ellos es la cruza, y su funci\'on es recombinar el
material gen\'etico. Se toman aleatoriamente dos individuos que hayan
sobrevivido al proceso de selecci\'on y se recombina su material gen\'etico
creando uno o m\'as descendientes, que pasan a la siguiente poblaci\'on. Este
operador se aplica tantas veces como sea necesario para formar la nueva
poblaci\'on.
El \'ultimo paso consiste en la aplicaci\'on del operador de mutaci\'on. Este
operador, que en general act\'ua con muy baja probabilidad, modifica algunos
genes del cromosoma, posibilitando de esta manera la b\'usqueda de soluciones
alternativas.

Una vez finalizado el proceso de selecci\'on, cruza y mutaci\'on se obtiene la
siguiente generaci\'on del algoritmo, la cual ser\'a evaluada, repiti\'endose el ciclo
descripto previamente. Tras cada iteraci\'on la calidad de la soluci\'on
generalmente va increment\'andose, y los individuos representan mejores
soluciones al problema.
Al algoritmo gen\'etico detallado anteriormente se lo conoce como algoritmo
gen\'etico can\'onico y es la forma m\'as utilizada. Sin embargo, en algunas
implementaciones particulares de algoritmos gen\'eticos se puede agregar
nuevos operadores. En todos los casos, la forma en que se implementen los
operadores variar\'a de acuerdo a las caracter\'isticas propias del problema. 
	\input{content/chapters/chapter5/cromosoma}

	%http://laboratorios.fi.uba.ar/lsi/bertona-tesisingenieriainformatica.pdf